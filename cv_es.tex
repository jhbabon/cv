%!TEX TS-program = xelatex
%!TEX encoding = UTF-8 Unicode

\documentclass[10pt, a4paper]{article}
\usepackage[spanish]{babel}

\usepackage{fontspec}

% DOCUMENT LAYOUT
\usepackage{geometry}
\geometry{
  a4paper,
  textwidth = 5.5in,
  textheight = 8.5in,
  marginparsep = 7pt,
  marginparwidth = .6in
}
\setlength\parindent{0in}

% FONTS
\usepackage[usenames,dvipsnames]{color}
\usepackage{xunicode}
\usepackage{xltxtra}
\defaultfontfeatures{Mapping = tex-text}
\setromanfont [Ligatures={Common}]{Palatino}
\setmonofont{Input}

% ---- CUSTOM COMMANDS
\chardef\&="E050
\newcommand{\html}[1]{\href{#1}{\scriptsize\textsc{[html]}}}
\newcommand{\pdf}[1]{\href{#1}{\scriptsize\textsc{[pdf]}}}
\newcommand{\doi}[1]{\href{#1}{\scriptsize\textsc{[doi]}}}
% ---- MARGIN YEARS
\usepackage{marginnote}
\newcommand{\years}[1]{\marginnote{\scriptsize #1}}
\renewcommand*{\raggedleftmarginnote}{}
\setlength{\marginparsep}{6pt}
\reversemarginpar

% HEADINGS
\usepackage{sectsty}
\usepackage[normalem]{ulem}
\sectionfont{\mdseries\upshape\Large}
\subsectionfont{\mdseries\large}
\subsubsectionfont{\mdseries\upshape\large}

% PDF SETUP
\usepackage[
  bookmarks,
  colorlinks,
  breaklinks,
  pdftitle = {Juan Hernández Babón - C.V.},
  pdfauthor = {Juan Hernández Babón}
]{hyperref}
\hypersetup{
  linkcolor = blue,
  citecolor = blue,
  filecolor = black,
  urlcolor  = blue
}


% DOCUMENT
\begin{document}
\reversemarginpar
{\LARGE Juan Hernández Babón}\\[0.25cm]
\href{mailto:juan.hernandez.babon@gmail.com}{juan.hernandez.babon@gmail.com}\\
\href{https://github.com/jhbabon}{https://github.com/jhbabon}\\

%%\hrule
\section*{Formación académica}

\textbf{Ingeniero Técnico en Informática de Sistemas}. Título obtenido en la Escuela Técnica Superior de Ingeniería Informática de la Universidad de Valladolid.

%%\hrule
\section*{Especialidades técnicas}

Estoy especializado en el desarrollo web, siendo mi área principal el backend. Aún así, tengo conocimientos en todo el entorno de desarrollo web, desde el frontend hasta la administración de servidores.\\
Tengo gran dominio sobre: Ruby (Rails), JavaScript (jQuery), HTML, CSS, bash, zsh, gestores de bases de datos (SQL: PostgreSQL, MySQL; NoSQL: Redis), servidores web (Apache, NginX).\\
Tengo competencia con: PHP, Python y C.\\
Trabajo y me desenvuelvo con comodidad en sistemas UNIX (Mac OS X y distribuciones de Linux). Mi editor principal es VIM, y uso Git como sistema de control de versiones para prácticamente todo.

\section*{Motivación}

Me importa el software. Siempre intento crear software que sirve de ayuda al usuario final, y es fácil de mantener por el resto de desarrolladores. Creo que esa es la idea central de todo desarrollo: crear software que es útil y fácil de usar a todos los niveles, no un problema. Sé que el mundo del desarrollo se mueve rápido, por eso siempre intento aprender todo lo que pueda sobre nuevas tecnologías y metodologías.

\section*{Experiencia profesional}

\noindent\years{2015 - hoy} Actualmente trabajo en \textbf{Redbooth} como desarrollador backend. También trabajo desde Barcelona. Soy parte de un equipo que se está a cargo de integrar otros servicios dentro de la app principal. Por ejemplo, integramos servicios como Dropbox o Evernote dentro de la app. El desarrollo se hace en Ruby con el stack propio de applicaciones Rails: colas de mensajes (Resque con Redis), servidores web (Ngnix con Unicorn), sistemas de caché, múltiples entornos de desarollo, etc.\\[.2cm]
\noindent\years{2013 - 2015} Trabajé en \textbf{Xing}, en Barcelona, como desarrollador backend. Fui parte de un equipo que estaba a cargo de tres applicaciones dentro de la compañía centradas en las ofertas de trabajo y publicidad de las compañías. Nuestro objetivo era la máxima visibilidad de las ofertas de trabajo para las personas que buscan empleo, tanto en Google como en \textbf{Xing}, así como de las compañías que las publicitan. El desarrollo se hace en Ruby con el stack propio de applicaciones Rails: colas de mensajes (RabbitMQ), sistemas de caché, múltiples entornos de desarollo, etc.\\[.2cm]
\noindent\years{2012 - 2013} Trabajé en \textbf{Wuaki.tv}, parte de \textbf{Rakuten}, en Barcelona, como desarrollador backend usando principalmente Ruby. \textbf{Wuaki.tv} es una app de streaming de videos, especializada en cine y series de tv. El mayor reto de este sistema es servir el contenido a la app web y a distintos dispositivos como TVs y tablets. En ella hemos tenido que resolver problemas de escalabilidad, gestión de entornos de desarrollo, despliegue de componentes, colas de mensajes, integración con otros sistemas y otros problemas relacionados con apps a gran escala.\\[.2cm]
\noindent\years{2011 - 2012} Desarrollador y jefe de proyectos en la empresa Gnuine, en Barcelona, usando Rails y el CMS open source Ubiquo. He participado en proyectos como la nueva web del \textbf{FC Barcelona}. He trabajado en otros proyectos como desarrollador Ruby, creando APIs de comunicación entre aplicaciones. También me he encargado del empaquetado de aplicaciones PhoneGap para Android y iOS, y su integración con los sistemas de notificaciones push de cada una de esas plataforma.\\[.2cm]
\noindent\years{2011} Desarrollador web en la empresa Alt120, en Barcelona. Uso del framework PHP Yii y jQuery.\\[.2cm]
\noindent\years{2009 - 2010} Encargado de los desarrollos web, en Ruby y PHP, y la administración de servidores Linux de la empresa IBECON 2003 S.L.\\[.2cm]
\noindent\years{2008 - 2009} Desarrollo de una aplicación web en el Proyecto de Fin de Carrera para la Fundación CARTIF. La aplicación se hizo con el framework PHP Symfony.

\begin{center}
{\scriptsize  Última actualización: \today\- •\- \href{https://raw.github.com/jhbabon/cv/master/cv_es.pdf}{Descargar la última versión de este documento}\- •\- Documento realizado con \href{http://nitens.org/taraborelli/cvtex}{ \fontspec{Palatino}\XeTeX }}
\end{center}

\end{document}
